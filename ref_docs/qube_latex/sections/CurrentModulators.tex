\section{Current Modulators}
Each channel has one current modulator designed to meet the most common needs while operating with optoelectronic devices and QCLs. Each modulator can be externally controlled with a analog signal ranging in the $\mathsf{\pm5 V}$ span. The V/I gain of the two modulators can be selected by properly connecting an on-board switch. \newline
The two different gains allow for the subsequent characteristics:
\begin{itemize}

\item\textbf{Low gain}: G = -1 mA/V, allows to add small modulating current signals to the DC bias, down to tens of nA, with a high level of precision and very low added noise.
\item\textbf{High gain}: G = -10 mA/V, allows to add modulating currents up to hundreds of mA to perform large laser frequency scans.

\end{itemize}
The modulators have a negative gain. If we denote the modulation current with $I_{mod}$, the input control voltage $V_{in}$ and the absolute value of the gain in $mA/V$ with $G$, the modulation current is equal to:

\begin{center}
    $I_{mod} = -G \cdot V_{in}$.
\end{center}

This means that a positive control voltage will subtract current from the laser bias current, while a negative control voltage will add to it.

\begin{tcolorbox}[enhanced,attach boxed title to top center={yshift=-3mm,yshifttext=-1mm},
  colback=black!5!white,colframe=red!75!black,colbacktitle=red!80!black,
  title=CAUTION,fonttitle=\bfseries,
  boxed title style={size=small,colframe=black!50!black} ]

The high current modulator can provide currents up to 100 mA which can be added or subtracted to the bias current of the laser.

While using negative control voltages the modulation current is added to the bias current, in this case the total current may exceed the maximum current allowed by the laser. Therefore it is necessary to be very careful to avoid the risk of damaging the laser itself.

Conversely, while using positive control voltages, no such risk occurs, since the modulation current is subtracted from the laser bias current.

\end{tcolorbox}